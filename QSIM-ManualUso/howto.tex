
\newcommand{\funAgregar}{Agregar archivos}
\newcommand{\funEnsamblar}{Ensamblar}
\newcommand{\funCargar}{Cargar en memoria}

\section{Ensamblar y cargar programas .qsim}
\subsection{Paso 0: Agregar archivos}
Como se observa en la figura \ref{ventana_cargado} de la ventana principal, se cuenta con la opción de \textbf{\funAgregar} .qsim y están deshabilitadas las otras opciones de \textbf{\funEnsamblar} o de \textbf{\funCargar}.
\graf{ventana_cargado}{Cargado de archivos}

\subsection{Paso 1: Ensamblar}
Como se observa en la figura \ref{pantalla_paso0_script} de la ventana principal, una vez que los programas Qi en los archivos .qsim se encuentran agregados se habilitan las opciones para:
\begin{itemize}
\item seleccionar la arquitectura Qi que se desee (Q1.. Q6)
\item ensamblar el programa para que se realice el chequeo de sintaxis y se genere el código máquina (que en el siguiente paso será cargado en memoria).
\end{itemize}  

\graf{pantalla_paso0_script}{Ensamblar}

Haciendo clic en el botón \textbf{\funEnsamblar}, se generará el código máquina correspondiente al código fuente de los archivos .qsim, en la arquitectura Qi elegida.

\subsection{Paso 2: Cargar en memoria}
Como se observa en la figura \ref{pantalla_paso1} de la ventana principal, se tiene habilitado el botón \textbf{\funCargar}. Haciendo clic en dicho botón, se abrirá una nueva ventana para visualizar el contenido de la memoria principal, donde se incluye el código máquina generado en el paso anterior.\\

\graf{pantalla_paso1}{Cargado en memoria}

Es importante notar que en esta etapa también están habilitadas las funciones de \textbf{\funAgregar} y \textbf{\funEnsamblar} ya que puede  agregarse otro archivo .qsim o quitar alguno (no todos) y volver a ensamblar. Además se puede determinar la celda a partir de la cual se quiere cargar en memoria el código máquina (por defecto se carga a partir de \code{0000}) y por lo tanto cual es el valor inicial del registro \code{PC}.

\newpage
%-------------------------------------
\section{Ejecutar programas .qsim}
Como se observa en la figura \ref{qsim_pantallaMemoriaSecciones} de la ventana de ejecución, una vez que el programa es cargado en memoria se abre esta nueva ventana que muestra:
\graf[1.2\textwidth]{qsim_pantallaMemoriaSecciones}{Ventana de ejecución}
\begin{enumerate}[(a)]
\item el mapa de memoria
\item el estado de los registros especiales \PC, \IR, \SP
\item el estado de los flags
\item el estado de los registros de uso general (R0...R7)
\item una consola de información al usuario
\item los botones para controlar la ejecución paso a paso
\item los botones para editar registros y ver los puertos
\end{enumerate}

Para poder realizar el ciclo de ejecución de una instrucción, se debe seguir en orden las etapas de búsqueda, decodificación y ejecución de la instrucción utilizando los botones del panel \textit{Ciclo de ejecución}.

\graf{pantalla_fetch}{Ventana de ejecución luego de la búsqueda}
\graf[0.5\textwidth]{consola_decode}{Detalle de la consola luego de la decodificación}
\graf[0.5\textwidth]{consola_exe1}{Detalle de la consola y los registros luego de la ejecución de la instrucción}

En las figuras \ref{pantalla_fetch}, \ref{consola_decode} y \ref{consola_exe1} se muestra el efecto de cada uno de los pasos. Al comenzar un nuevo ciclo de ejecución, las celdas leídas se muestran en otro color, como se muestra en la figura \ref{detalle_fetch2}.


\graf[0.7\textwidth]{detalle_fetch2}{Detalle de los registros especiales y parte de la memoria luego de la búsqueda de la siguiente instrucción}

\subsection{Visualización de puertos de Entrada/Salida}

Si se desea administrar el valor de los puertos, se debe hacer clic en el botón \textit{Ver puertos}, y se abrirá una nueva ventana como se muestra en la figura \ref{pantalla_puertos}. 
\graf[6cm]{pantalla_puertos}{Ventana de edición de puertos de Entrada/Salida}\\

