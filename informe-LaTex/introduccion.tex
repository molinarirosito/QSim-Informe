\chapter*{Resumen}

\parbox{8cm}{
Una de las primeras asignaturas que debe recorrer un estudiante de la \tpi\ es \textbf{\orga}. En esta materia los estudiantes descubren los componentes funcionales que conforman un sistema de cómputos, con el fin de comprender un modelo de ejecución de programas que está presente hoy en día en la mayoría de las computadoras personales.\\

Este trabajo es el desarrollo de una herramienta que permite simular la ejecución de programas en una arquitectura teórica desarrollada por el equipo docente de la materia.\\

En este documento se presenta primero información contextual que incluye conceptos importantes y el enfoque de la materia. Luego se detalla la funcionalidad del simulador y como es el diseño orientado a objetos, siguiendo por un apartado donde se realiza un análisis del desarrollo, incluyendo el diseño de los casos de prueba. Finalmente se incluyen anexos con detalles de la arquitectura \Q, y los errores comunes en el uso del simulador.
}
%\end{abstract}